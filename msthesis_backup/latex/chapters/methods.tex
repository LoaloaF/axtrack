\section{Methods and Materials}

\subsection{Experimental procedures}

\subsubsection{Dish preparation}
First, WillCo glass dishes ($\O$30 mm, WillCO Wells) were rinsed with acetone,
isopropanol and ultra pure Water (Millipore Milli-Q System, 18M$\Omega$), then
dried with a nitrogen gun. Next, double sided adhesive (DSA) rings were used to
attach WillCo glass dishes to polystyrene dish frames. The assembled dishes were
placed in a larger plastic dish with tape stripes preventing surface adhesion
between dishes.

% respectively, and dried with nitrogen gun. Finally, the cover slip
% was sealed off to the bottom of the dish. For the experimental groups, small
% plastic Petri dishes ($\O$40 mm) were coated with PDMS by adding approximately
% 250 $\mu$L of PDMS to the centre of the dish without forming any bubbles,
% spin-coating at 1500 rpm for 60 secs and then curing for 2 hours at 80
% $^{\circ}$C. The dishes were placed inside a bigger Petri dish to make the
% handling easier. A strip of tape was added to the bottom of the big Petri dish
% in order to prevent glass bottom dishes sticking. Prior to coating, all the
% dishes were sterilised under UV light for 4 hours or 2 hours at 80 $^{\circ}$C.

\subsubsection{Poly-D-Lysine \& laminin coating}
The assembled glass dishes were coated using 1 ml Poly-D-Lysine (PDL) solution,
incubating for 1-2 hours at room temperature. The solution was prepared using 1
ml of thawed up PDL stock (P7280, Sigma-Aldrich), and 8 ml of phosphate buffered
saline (PBS) (10010015, Gibco, Thermo Fisher Scientific, Switzerland). After
incubation, the PDL solution was removed and the dishes were washed 2 times with
PBS, and once with deionized (DI) water. \\

% PDL coating solution  was prepared by adding 1 mL of PDL (P7280, Sigma-Aldrich)
% stock was thawed and mixed with 8 mL of sterile PBS (10010015, Gibco, Thermo
% Fisher Scientific, Switzerland). 2 mL of the PDL solution was added to each dish
% and incubated at 4 $^{\circ}$C over night. The PDL solution was washed 3x with
% PBS and then once 1x with sterile DI water. PDL solution should be washed
% completely since non-physisorbed PDL polymer fragments can be toxic for the
% neurons. \cite{shin2012microfluidic} \cite{lau2013cell}) Finally, the dishes
% were dried in the hood for 1 hr. 

Subsequent to PDL application, dishes were coated with 10 $\upmu$g/ml laminin.
This solution was prepared by slowly thawing 50 $\upmu$l aliquots on ice, then
adding 5 ml of Neurobasal$\rm^{TM}$ plus (A3582901, Gibco). Between 300-800
$\upmu$l laminin solution was applied to cover the whole surface of the glass
dish. After 24h incubation at 37 $^{\circ}$C, laminin solution was removed and
the dishes were washed 1 time with PBS, and 2 times with DI water.


% In order to prepare the laminin coating solution of concentration 10 $\mu$g/mL,
% 50 $\mu$L laminin (1 mg/mL) stock was thawed on ice to prevent gelation and was
% later added to 5 mL Neurobasal$^{TM}$ plus medium (A3582901, Gibco). For
% experimental groups involving laminin coating, the surface of dish was covered 2
% mL with the laminin solution after the PDL washes and incubated over night at 37
% $^{\circ}$C incubator or at 4 $^{\circ}$C for 48 hours. After the incubation,
% laminin solution was washed 1x with PBS and 2x with sterile DI water to avoid
% formation of salt crystals.


\subsubsection{PDMS micro structure design, fabrication \& mounting}
\label{pdms structures assembly}
PDMS micro structures were designed in a multi stage computer aided design (CAD)
process. This was necessitated by the vast number of design motifs investigated
in this study. Although not primarily intended for 2D CAD, Fusion360 (Autodesk,
San Rafael, California) was used in the initial design stage. Fusion360 was
chosen because of its powerful version control and design history system,
enabling the natural integration of design variables into the CAD workflow. More
concretely, specific elements in different PDMS designs were inserted as
separate components such that they could be updated independently from the base
designs; for example the commonly used 2-joint motif. Single PDMS designs were
then exported as \verb|.dxf| files by projecting extruded bodies to 2D sketches.
Importantly, the projection link had to be deleted to export valid \verb|.dxf|
files. The single PDMS designs were imported to AutoCAD (Autodesk, San Rafael,
California) to define fabrication mask layers and arrange designs on the wafer.
Finally, the wafer design was exported as a \verb|.dxf| file and imported from
KLayout where the final \verb|.gds2| file was generated. The wafer and PDMS
designs were fabricated by Wunderlichips (Switzerland) employing standard soft
lithography (for details see \cite{forro}). \\

The PDMS membrane delivered by Wunderlichips was separated into independent PDMS
structures on a laser cutter (Speedy300, Trotec, Switzerland) using 8 \% power
at 14 cm/s. Subsequently, the structures were thoroughly rinsed with 70 \%
ethanol. For a subset of experiments, a PDMS frame was laser cut from 5mm thick
cured PDMS. Using uncured PDMS, it was attached on the micro structure to
enclose the output channel area. After curing for 1h at 80$^{\circ}$C, they were
picked up with a pair of surgical forceps and slowly placed on the coated glass
dishes (see above). Importantly, a thin film of DI water was put on the glass
dishes to facilitate mounting without enclosed air bubbles. As the last dish
preparation step, RGC medium (see composition in Table \ref{rgcmedium}) was
added and the dishes were desiccated for 30-60 minutes to remove air from the
PDMS micro channels.

\begin{table*}
    \begin{adjustbox}{max width=\textwidth,center}
        \begin{tabular}{@{}rrrrrrrr@{}}
            \toprule
            & & & & & Component & Volume [ml] & Stored at [$^{\circ}$C] \\
            \midrule
            & & & & & Neurobasal Plus (Gibco, A3582901)               & 237.5    &   4\\
            & & & & & DMEM (Gibco 11960                               & 237.5    &   4\\
            & & & & & Glutamax                                        & 5        &   4\\
            & & & & & Sodium Pyruvate (100mM, Gibco 11360-070)        & 5        &   4\\
            & & & & & Antibiotic-Antimycotic (100x, Gibco 15240096)   & 5        & -20\\
            & & & & & N2 Supplement                                   & 5        & -20\\
            & & & & & B27+ (50x)                                      & 10       & -20\\
            & & & & & N21 Supplement (50x, R$\&$D Systems AR008)      & 10       & -20\\
            & & & & & NAC Stock (5 mg/mL)                             & 0.5      & -20\\
            & & & & & Forskolin Stock (4.2 mg/mL)                     & 0.5      & -20\\
            & & & & & BDNF Stock (50 $\mu$g/mL, Preprotech 450-02)    & 0.5      & -20\\
            & & & & & CNTF Stock (10 $\mu$g/mL, Preprotech 450-13)    & 0.5      & -80\\
            & & & & & NGF 7S Stock (10 $\mu$g/mL, final 10 ng/mL)     & 0.5      & -80\\
            & & & & & GNDF (10 ng/mL)                                 & 0.5      & -20\\
            \bottomrule
        \end{tabular}
    \end{adjustbox}
    \caption[RGC medium composition]{RGC medium composition. This medium was
            used throughout for culturing RGC neurons.}
    \label{rgcmedium}
\end{table*}

% Briefly, a negative SU8 photoresist (SU8 3000 Series, MicroChem, USA) was spin
% coated on a silicon wafer, selectively cross-linked by passing UV light through
% a photomask in two consecutive layers, and the uncrosslinked photoresist
% dissolved in mr-Dev 600 (Microresist Technologies, Germany) to create a mold for
% the PDMS devices. Next, the mold was treated with perfluoro-octyl
% trichlorosilane (AB111444, ABCR, Germany) in a desiccator chamber with low
% vacuum applied. PDMS (Sylgard 184, Dow Corning) was then spin coated, cured
% overnight at 70 °C , peeled off and transferred to a clean container.
% Uncrosslinked PDMS was extracted from the devices by immersing them in iso-
% propanol for a minimum 1 h to improve cell viability (Millet et al., 2007).


\subsubsection{Spheroid creation}
pass



\subsubsection{Primary Cell Culture (EYLUL)}
All cell culture experiments were performed using primary cells from cortices
and eyeballs of E18 embryos of time-mated pregnant rats (Janvier Laboratories,
France). Animal experiments were approved by the Cantonal Veterinary Office
Zurich.


\subsubsection{Retina Dissections (EYLUL)} 
 Dissection instruments, microscalpels, scissors and forceps were sprayed with
 70 $\%$ ethanol prior to dissections. The retina dissections were performed
 under a benchtop microscope (DFC420C with 4X magnification, Leica, Germany) in
 a Petri dish filled with hibernate medium. Retinas were dissected out from
 whole eyeball. Firstly, all the tissue around the eyeballs was removed. The
 eyeballs were pinched along the cornea-sclera edge and cornea with forceps on
 both sides and gently pulled apart to cut open and isolate the retina. After
 gently removing the lens, the retina was later cut into square explants of
 around size 500 $\mu$m X 500 $\mu$m. After the dissections, the ~100 $\mu$L
 dissected retina explants were transferred to a small Eppendorf tube and tagged
 with an adeno-associated virus (AVV) encoding for the mRuby virus
 (scAAV-DJ/2-hSyn1-chl-mRuby3-SV40p(A)). To do so, mRuby virus vial was thawed
 on ice and 1 $\mu$L was added to the explants. The explants were incubated with
 mRuby on ice for 1 hour.


\subsubsection{AggreWell\textsuperscript{\texttrademark} Preparation and Cell Dissociation (EYLUL)}
AggreWell\textsuperscript{\texttrademark} plate preparations to produce
reproducible spheroids and cell dissociation were performed in parallel.
AggreWell\textsuperscript{\texttrademark} 800 microwell culture plates were
prepared by adding 500 $\mu$L of AggreWell\textsuperscript{\texttrademark}
rinsing solution to the needed wells in order to prevent cell adhesion and
promote spheroid formation. The plate was then balanced by adding 300 $\mu$L of
DI water to each well of a standard well plate and centrifuged at 2000 x g for 5
minutes. The plate was examined under the microscope and check for bubbles. If
there are trapped bubbles in the micro-well, the centrifuge procedure was
repeated again. Afterwards, the AggreWell\textsuperscript{\texttrademark}
rinsing solution was aspirated and each well was rinsed with 2 mL of warm
Neurobasal medium. 1 mL of complete medium was added to each well and the plate
was kept in the incubator until cell dissociation was completed. For the cell
dissociation, firstly, PBG solution was prepared by mixing 50 mg BSA in 50 ml
sterile PBS together with 90.08 mg glucose. In order to prepare the Papain
solution, 2.5 mg Papain was added to 5 mL of PBG and vortexed. After allowing 30
mins for dissolving, the solution was sterile filtered using a 0.2 $\mu$m
filter. Finally, 5 $\mu$l DNAse was added. 5 ml Papain solution was added, mixed
gently and incubated at 37 $^{\circ}$C for 15min and was shaken gently every
5min. Papain solution was aspirated without disturbing the pellet and 5ml
Neurobasal media supplemented with 10 $\%$ FBS was added. After waiting for 3
min, media was removed without disturbing the pellet. This wash step was
repeated twice by adding 5 ml Neurobasal media, waiting 3 minutes and removing
medium. Finally, ~4mL Neurobasal Plus medium supplemented with B27+ and
anti-anti was added for ~8 cortices. The cortices were then pipetted up with a 5
mL Pipette boy and quickly ejected to dissociate the cells. The cells were
strained using a 40 $\mu$m cell strainer and a cell count was performed using
Trypan Blue and a hemocytometer. Once the viable cell concentration was
determined, concentration of the cell suspension was adjusted to determine the
number of cells required to obtain 8000 cells per microwell based on the desired
number of cells per microwell multiplied by 300 microwells per well. After
adding the required volume of the cell suspension to the wells, complete medium
was added to each well to have a final volume of 2 mL in each well. 1 $\mu$L
mRuby and with 0.5 $\mu$L of the calcium indicator was added into each well(s)
to transduce the cells. The medium in the wells were pipetted up and down to
make sure cells were evenly distributed.
AggreWell\textsuperscript{\texttrademark} plate was balanced again and
immediately centrifuged at 100 x g for 3 minutes to ensure that the cells were
captured in the micro-wells. Even distribution of cells inside the micro-wells
were confirmed under the microscope and the plate was incubated at 37
$^{\circ}$C with 5$\%$ CO$_{2}$ for 24 hours before seeding to allow for the
formation of spheroids. 

\subsubsection{Cell Seeding and Primary Neuronal Cell Culture (EYLUL)}
Explant and spheroid seeding into the wells were performed under a benchtop
microscope (DFC420C with 4X magnification, Leica, Germany). Virus tagged
explants and spheroids were gently pipetted and transferred to the wells of the
mounted structures. Microscalpels were later used to insert the explants or
spheroids into the wells. After the seeding, a small petri dish ($\O$40 mm) was
filled with 2 mL of sterile DI water supplemented with 5 $\%$ anti-anti and
placed next to the culture dish in the big dish in order to minimise evaporation
of the cell culture medium and prevent drying out of the structures. For the
electrodes, a cover was placed on top of the PDMS ring. Finally, cell culture
dishes were transferred to the incubator and kept undisturbed for 3 days at 37
$^{\circ}$C , 5 $\%$ CO$_{2}$. A half-medium change was performed every 2 days
by using RGC medium for retina explants and Neurobasal plus medium supplemented
with B27+ and anti-anti for cortical spheroids.

\subsubsection{Timelapse recording}
Fast solid state storage media was used to prevent a frame rate bottleneck from 
an insufficiently fast network connection.










\subsection{Data analysis}
\subsubsection{Timelapse datasets}
This work incorporates three PDMS micros structure timelapse recordings that were
acquired at different timepoints. Accordingly, the data used in this project is
based on three experiments, where each experiment was performed with 8-14 rat
embryos (compare \ref{} for details). One of the three timelapses was solely
obtained for generating model training data, thus the presented results are
based on two experiments with a total of 16-28 biological replicates. A summery
of timelapse datasets is given in Table \ref{datasets_table}. 

\begin{table*}
\begin{adjustbox}{max width=\textwidth,center}
    \begin{tabular}{@{}rrrrrrrrrrrr@{}}
    \toprule
    & Acquired & Setup  & T [min] & n frames & Length [days] & Model usage\\
    \midrule
    \vspace{2mm}
    Dataset1 & 20.12.20 & 1 designs  &  40 & 37  & 1   & Training\\
    Dataset2 & 27.08.21 & 2 designs, chamber  &  31 & 210 & 4.5 & Training\\ 
    \vspace{2mm}
    Dataset3 & 27.10.21 & 18 designs, chamber &  31 & 210 & 4.5 & Inference\\
    Dataset4 & 07.10.21 & 21 designs, stomachs &  32 & 242 & 5.4 & Inference\\
    \bottomrule
    \end{tabular}
\end{adjustbox}
\caption[Overview of timelapse recording data]
        {Overview of timelapse recording data.
         \textit{n designs} refers to the number of unique PDMS micro structure
         designs composing the dataset. \textit{Chamber} setups employed large  
         thalamic tissue pieces enclosed by a PDMS frame for concentrated
         attraction cues (see \ref{pdms structures assembly} for details).
         \textit{Stomach} setups omitted PDMS frames and instead seeded a
         thalamic spheroid in the target well (see Figure \ref{R_designs} for
         stomach illustration). T refers to the temporal period of the
         recording. White space between rows indicates different experiments.}
\label{datasets_table}
\end{table*}

\subsubsection{Initial timelapse processing}
The proprietary \verb|.oir| files produced by the CLSM were converted to three
dimensional \verb|.tif| files using Python's \verb|bioformats| package, which
relies on a java virtual machine implemented within the \verb|javabridge|
package. Additionally, \verb|.tif| frame sequences were rendered to \verb|.mp4|
video using \verb|scikit-image| and \verb|open-cv| (Suppl. Figure
\ref{SF_intial_tl_preprocessing} A). These videos were used for initial
evaluation of the timelapse, validating for example absence of undergrowth. The
transmission channel of each PDMS micro structure timelapse was then loaded into
napari, a python based n-dimensional image viewer \parencite{Sofroniew2021}.
Using \verb|skimage.filters| to segment the micro channels in the PDMS designs,
edge magnitude was detected with \verb|prewitt()|, gaussian smoothing was
performed with \verb|gaussion()| ($\rm\sigma=1$), thresholding was done with
\verb|threshold_otsu()|, and finally, the segmentation was cleaned up using
\verb|skimage.morphology.binary_closing()| (diameter = 4) (Suppl. Figure
\ref{SF_intial_tl_preprocessing} B). Subsequently, the segmentation of the PDMS
micro channels was manually cleaned up, mainly using the bucket tool to fill
areas enclosed by detected edges. To remove patches in the mask, the target
point of the PDMS design was labelled and used as the origin to perform
\verb|skimage.segmentation.flood()|. As a last step, both the final output
channel and the first 100 $\rm\upmu m$ of the channels exiting the source wells
were segmented and saved as binary masks (Suppl. Figure
\ref{SF_intial_tl_preprocessing} C).

\subsubsection{Axon growth cone labelling}
The axon growth cone tracking model was trained using Dataset1, and Dataset2
which included timelapse recordings of three unique PDMS designs (compare Table
\ref{datasets_table}). The labelling of these three image sequences was
performed by one human expert using the napari image viewer (Suppl. Figure
\ref{SF_labelling} A). The workflow for obtaining the four dimensional label of
\verb|FrameID| - \verb|AxonID| - \verb|X coordinates| - \verb|Y coordinate| was
as follows:
\vspace{2mm}
\begin{enumerate}
    \item Load timelapse sequence.
    \item Create empty set of axon identities.
    \item Inspect short time slice of 3-6 frames for distinct, coherently moving
    blob.
    \item Identify axon identity by its growth cone.
    \item Trace axon identify over adjacent frames until unidentifiable.
\end{enumerate}
\vspace{2mm}
In the scenario where two separate growth cones converge forming a single
observable growth cone, one of the two identities was arbitrarily chosen to be
continued while the other one was terminated. Hence, the underlying number of
axons for a given growth cone label may be larger than one. It should also be
considered that there is some degree of uncertainly in the ground truth labels.
Especially when the PDMS micro channels become largely filled, distinguishing
between GFP-protein trafficking along existing axons versus new growth cones
becomes challenging. The annotations here were consistently done more
conservatively, weighting the avoidance of false positives higher than missing
true positives. Following this conservative labelling methodology, an axon
identity was only considered if it appeared over more than three frames. From
three concatenated PDMS micro structure timelapses, 300 growth cones were
identified over N=327 frames where the average axon identity lifetime was 24
frames. An overview of the identify lifetime is given in Suppl. Figure
\ref{SF_labelling} B; four labelled example frames are shown in Suppl. Figure
\ref{SF_training_data}.

\subsubsection{Timelapse data preprocessing}
The CLSM \verb|12bit| gray scale intensity values saved as 
\verb|16bit unsigned integers| were first converted to a scale of 0 to 1 using 
\verb|skimage.util.img_as_float()|. For image sequences that had an offset in 
the intensity profile, this offset was subtracted such that the minimal 
intensity was always 0. Next, the segmentation of the micro channels 
was used to mask the image sequence (see Suppl. Figure 
\ref{SF_intial_tl_preprocessing} C for example mask). The resulting initial 
distribution of intensity values for both training and inference data is shown 
in  (Figure \ref{M_preprocessing} A top left). In the next step, intensity 
values below threshold = 0.00083 were clipped and set to 0 (Figure 
\ref{M_preprocessing} A top right). Next, the intensity profile $I_{in}$ was 
stretched using \verb|skimage.exposure.adjust_log()| function with a 
gain  \textit{g} = 1 which transformed the distribution according to formula 
\ref{formula_log_adjust} (Figure \ref{M_preprocessing} A bottom left). 

\begin{equation}
    I_{out} = g*log(1 + I_{in})
    \label{formula_log_adjust}
\end{equation}

Finally, the intensity distribution was divided by the global standard deviation
across the entire training image sequence, ensuring unit variance in the model
input data (Figure \ref{M_preprocessing} A bottom right). Both frame-wise, and
mean-related standardizations were omitted since their application resulted in
decreased detection performance. The intensity distributions from train- and
inference data do not overlap in Figure \ref{M_preprocessing} A because
the sparsity differs vastly across frames. Train intensity values do not
increase from $\rm t_0$ to $\rm t_N$ because $\rm t_N$ corresponds to a
different timelapse video (Dataset2) which is more sparse than $\rm t_0$
(Dataset1).

\begin{figure}
    \includegraphics{M_preprocessing.pdf}
    \caption[Growth cone detection model overview]{Growth cone detection model
             overview. \textbf{A} Pixel intensity distribution of training-, and
             inference data at $\rm t_0$ and $\rm t_N$ over three major
             preprocessing steps. The two plots top left show the initial pixel
             intensity distributions, top right after clipping, bottom left
             after log adjusting, and bottom right after standardization.
             Histograms on the left were obtained from sampling $10^6$ pixel
             values from the first frame, histograms on the right from the last
             respective frame. The inference data to produce the histograms
             (brown) was taken from a representative image sequence from
             Dataset3 (see Table \ref{datasets_table}). The number in the legend
             indicates the proportion of image values equal to zero (sparsity).
             Note that outlier intensity values are not shown. \textbf{B} CNN
             architecture. Each yellow block in represents a sequence of
             2D-convolution, batch normalization \parencite{BN}, and Leaky ReLU
             \parencite{leakyrelu}. Orange layers stand for maximum pooling
             operations. FC stands for fully connected layers. \textbf{C}
             Example tile illustrating the YOLO label format. Each grid box can
             be predicted to contain a growth cone. In this example, 4 of 12 x
             12 grid boxes are positive. Green colors in the tile represent
             positive motion (pixel intensity increased from frame $\rm t_0$ to
             $\rm t_1$), blue represents negative motion. Grid box size = 26
             $\rm \upmu m$.
             
             } 
    \label{M_preprocessing}
\end{figure}

\subsubsection{Growth cone detection model}
\label{modeldescription}
\paragraph{Temporal context frames}
The growth cone detection model implemented in \verb|PyTorch| follows the
general approach of YOLO (You Only Look Once) \parencite{yolo} where the
detections are obtained by a single pass through the network (Figure
\ref{M_preprocessing} B). The first aspect in which it deviates from the
original is that instead of inputting an RGB image, the network receives a
temporal stack of five gray scale images. Concretely, to detect growth cones at
frame $\rm t_0$, frame $\rm t_{-2}, t_{-1}, t_{0}, t_{1}, t_{2}$ are fed into
the network. This architecture aims to imitate the strategy of human labelling:
by inspecting single frames, growth cone identification is highly uncertain;
only when scanning sequences of frames, coherently moving blobs of particular
shape and dynamics can be linked to growth cones and thus an axon identity. To
always provide full temporal context, frame $t_{0}, t_{1}, t_{N-1}, t_{N}$ were
omitted from detecting growth cones in the image sequence. Computing the motion
between frames manually by subtraction yielded decreased detection performance
over the implicit approach of passing temporal context frames. An illustration
of the motion computation is shown in Figure \ref{M_preprocessing} C.

\paragraph{Tiling}
Each yellow block in Figure \ref{M_preprocessing} B represents a sequence of
2D-convolution, batch normalization \parencite{BN}, and Leaky ReLU
\parencite{leakyrelu}. Orange layers stand for maximum pooling operations. As
performing detection on the original resolution of 3868 x 1972 was
computationally intractable, the timelapse frames were split into 512 x 512
tiles (see Figure \ref{M_preprocessing} C and grid in Suppl. Figure
\ref{SF_training_data}). The CNN computes a convolutional feature map of 16 x 16
x 160, thus a single \textit{feature pixel} represents a region of $\rm
\frac{512}{16}$ = 32 pixels in the original 512 x 512 input image. The CNN
output resolution was a relevant consideration for its architecture, as the
detection objects of interest are small and potentially locally clustered (using
microscopy settings described in \ref{}, growth cones are between 4-26 pixels).
If the same CNN feature output resolution was to be achieved using original
timelapse frames, the CNN output would be of shape of $\rm \frac{3868}{32}$ x
$\frac{1972}{32}$ x $160 \approx 120$ x $61$ x 160. Storing the weights between
this high-resolution CNN feature map and the first fully connected layer exceeded  
GPU memory. An additional computational benefit is achieved by skipping empty
tiles. From visual inspection, the discontinuities between tiles did not seem to
result in decreased detection performance for growth cones near the tile edges. 

\paragraph{Detection output format}
Following the general YOLO label format, the network is trained to find a
mapping from a single tile CNN feature map to a 12 x 12 x 3 array. Here, the
first two dimensions represent a grid of the input tile, the last dimension
refers to the confidence of the respective grid box containing a growth cone,
and X-, Y grid box coordinates referring to the relative location of a growth
cone within the box (Figure \ref{M_preprocessing} C). This representation
results in the limitation, that only one growth cone can be detected per grid
box. As the example in Figure \ref{M_preprocessing} C shows, close growth cones
may still be detected as two separate identities if their centers are located in
different grid boxes. In the worst case scenario, the spatial detection
resolution of multiple growth cones is limited by the grid box size which is
equal to $\frac{512}{12} = 43$  pixels or 26 $\rm \upmu m$. This resolution was
sufficient for the application of our model as densely grouped growth cones were
the exception. \\
To drop overlapping detections, non max suppression was applied to the final
detection output according to \parencite{nms} using a minimum euclidean distance
of 23 pixels.

\paragraph{Training procedure}
The training data was split into 287 train frames (0.87), and 40 (0.13)
consecutive test frames which spanned two different PDMS micro structures. The
final model used for inference was trained on the entire dataset. Using
translation, rotation, horizontal and vertical flipping as data augmentation,
the model was trained up to convergence for 1000 epochs (Figure
\ref{R_modelresults} A). The loss function below (\ref{lossfunction}) is a
slight modification from the original.

\begin{equation}
    \lambda_{anchor}\sum_{i=0}^{S^2}\sum_{j=0}^B \mathbb{I}_{ij}^{obj}[(x_i-\hat{x}_i)^2 + (y_i-\hat{y}_i)^2 ] \\
    + \lambda_{obj}\sum_{i=0}^{S^2}\sum_{j=0}^B \mathbb{I}_{ij}^{obj}(c_i - \hat{c}_i)^2 \\
    + \lambda_{noobj}\sum_{i=0}^{S^2}\sum_{j=0}^B \mathbb{I}_{ij}^{noobj}(c_i - \hat{c}_i)^2 \\
    \label{lossfunction}
\end{equation}

\vspace{3mm}
\noindent
where $S = 12$ is the number of tiles, $B = 1$ is the number of detections per
grid box, $\mathbb{I}_{ij}^{obj}$ equals to 1 if a growth cone exists, 0
otherwise, $\mathbb{I}_{ij}^{noobj}$ equals to 0 if a growth cone exists, 1
otherwise, $c_i$ refers to the confidence that an object exists in the grid box,
and $x_i, y_i$ represent the grid box coordinates. $\hat{.}$ stands for the
ground truth label. The loss terms for predicting coordinates, object presence,
and object absence are weighted according to $\lambda_{anchor} = 45$,
$\lambda_{obj} = 54.25$, and $\lambda_{noobj} = 0.75$ respectively. The
balancing of those terms is based on the proportion of positive grid boxes which
is $\approx 0.7\%$. The initial learning rate was set to 0.0005 and decayed
with a rate given in formula \ref{lrdecay}.

\begin{equation}
    \gamma = e^{-\frac{1}{10}\sqrt{x}}
    \label{lrdecay}
\end{equation}

\noindent
where $\gamma$ is multiplied with the original learning rate, and x refers to
the current epoch. \verb|Pytorch|'s Adam optimizer \parencite{adam} was used
for fitting the model with $\beta_1 = 0.9$, $\beta_2 = 0.999$.

\paragraph{Data assocication}
\label{data_association}
The implemented growth cone tracking model follows a classical object tracking
paradigm of splitting the problem into object detection and identity
association. For this second step, the detections produced by the YOLO-like
architecture need to be classified into unique growth cone identities that live
over several video frames. In this work, identity assignment is framed as a
graph problem where we seek minimum cost flow solutions \parencite{MCF}. At a
high level, nodes represent detections at particular frames, and edges represent
identity associations between them (see illustration in Figure
\ref{R_modelresults} E). Each detection confidence is also represented as an
edge, which elegantly incorporates the detection model uncertainty into forming
identity trajectories through the graph and avoids the setting of explicit
detection confidence thresholds. At the basis of associating detections between
frames is the cost we assign between them. This cost can be interpreted as the
likelihood the two detections correspond to the same growth cone identity. In
contrast to other domains where visual similarity is highly relevant, the edge
costs here are completely based on the spatial distance between detections.
Specifically, the A* distances \parencite{astar} between them computed on the
segmented PDMS micro channel mask. This cost considers the constraint, that
growth cones can only translocate within the micro channels and are limited in
outgrowth speed. Solving the graph optimization problem includes the constraint,
that a node can only receive and emit a single edge, or in other words, a node
can only represent a single identity. As shown by \cite{MCF}, the graph can be
solved optimally and efficiently using integer linear programming. The
implementation here is based on the open-source package \verb|libmot|, using the
build-in \verb|MinCostFlowTracker|. The optimal hyperparameters listed and
explained in table \ref{MCF_params} were identified with a grid-search
algorithm on test data (see Figure \ref{R_modelresults} D). 




\begin{table*}
    \begin{adjustbox}{max width=\textwidth,center}
        \begin{tabular}{@{}rrrrr@{}}
            \toprule
            Edge cost threshold      & Entry-exit cost           & Miss rate            &  &  \\
            0.7                      & 2                         & 0.6                  &  &  \vspace{5mm} \\
            \toprule
            Maximum number of misses & Minimum network flow      & Maximum network flow &  &  \\ 
            1                        & 5                         & 450                  &  &  \vspace{5mm} \\
            \toprule
            Visual similarity weight & Confidence capping method &                      &  &  \\
            0                        & \textit{scale}            &                      &  & \\
        \end{tabular}
    \end{adjustbox}
\caption[Minimum cost flow hyperparameters]
    {Minimum cost flow hyperparameters. The edge cost threshold determines if an
    edge is pruned or kept, the entry-exit cost defines the cost of creating and
    terminating identities, the maximum number of misses indicates for how many
    frames an identity can be not detected, but still not terminated, the miss
    rate determines how much cost is incurred from missing detections (low means
    high cost), minimum and maximum network flow gives the minimum and maximum
    number of identities over all frames, visual similarity weight determines
    the degree to which visual similarity between detections contributes to the
    cost, and finally the confidence capping method sets the behavior for
    confidence values above 1, where \textit{scale} means normalize to maximum
    confidence.}
\label{MCF_params}
\end{table*}



\subsubsection{Directionality inference from tracking}
Using the growth cone trajectories obtained from the model, the degree of
directional growth through the PDMS micro structures can be inferred. Analogous
to the identity association cost employed for formulating the graph,
directionality inference also relies on computing A* paths on the micro channel
segmentation mask. For each detection that composes the growth cone trajectory,
we compute the shortest distance to a human-labelled target point located in the
output channel (see Suppl. Figure \ref{SF_intial_tl_preprocessing} C). Our PDMS
designs have the property, that the shortest path towards this point is
precisely the desired growth path. More intuitively, the desired growth path
towards the output channel has no detours. Thus, we can employ the A*-shortest
path towards the output channel as a proxy for confirming the correct growth
direction through the micro channels. Concretely, a correctly growing axon
should exhibit constantly decreasing A* distances towards the output channel.
Conversely, any increase in distance indicates growth in the undesired
direction.

\subsubsection{Directionality evaluation metric}
A variety of metrics could be considered for evaluating the degree of
directional growth through PDMS micro structures. At the basis, metrics rely on
the distance towards the output channel as explained above. An example of these
distances in a particular timelapse is shown in ??.  We considered comparing the
number of axons with negative,- and positive average slopes, as this would
indicate the number of  correctly,- versus incorrectly growing axons,
respectively. \\

We need to consider the biases of each metric. Given that structures join at
different locations, you get an inerent advantage for those that join late. \\


growth speed is lower in more complicated designs, benefit from the mtric above \\


metric that combines reached target, reached neighbour and directionality? 
Correct dir 1 point, reached goal 5? Arbitrary weighting.... \\

median vs mean gives very different story. argue for mean as outliers matter \\

v5 has reached neighbour, reached target, v6 has directions data \\

Speed and stagnation results 

We chose metric...


\subsubsection{Statistics}

\subsubsection{Between dataset variance}
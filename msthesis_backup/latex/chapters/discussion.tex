\section{Discussion}

\subsection{Summary}
This work set out to identify PDMS micro structures that directionally connect
multiple source nodes with a common target. Solving this problem would not only
be highly beneficial for biohybrid brain interfacing technology as described in
the introduction, but also for functionally more sophisticated neural network
models. To screen the 21 previously designed PDMS micro structures at high
throughput, we resorted to a computationally aided anatomical readout. Our
deep-learning based growth cone tracking model is able to accurately detect
growth cones of various morphologies and successfully link them to long living
axon identities. The obtained axon tracks allowed us to derive information on
the degree of directional growth, growth velocity and the number of outgrowing
axons. By virtue of the systematically parameterized 21 designs, we were further
able to identify trends between these metrics and specific design features. \\

Compared to 8 $\upmu$m wide channels, we find that narrow channel widths such as
1.5 and 3 $\upmu$m eliminate axon stagnation and significantly increase axonal
growth velocity by almost two folds. Conversely, introducing more sophisticated
design motifs such as 2-joints or rescue loops reduced growth velocity, with the
number of 2-joints additionally reducing the frequency of axonal outgrowth. When
comparing the relative proportion of backwards growing axons across the 21
designs, we find a significant relationship, but forward growth does not differ
significantly. Various features such as channel width, the number of 2-joints
and the number of rescue loops significantly affect backwards directionality.
Forward growth differs in designs varying in 2-joint placement and number of
rescue loops. Next, to find the best performing design and assign an
interpretable directionality metric, we calculated a ratio of target-grown and
cross-grown axons. With a fold directionality between almost 4 and 2.5, this
metric identified Design 20, 14, 13, 19 and 18 implementing one 2-joint, three
rescue loops and altered final lanes and 2-joints as the best performing ones. \\

While the tracking of growth cones through PDMS micro structures revealed highly
directional designs, the experimental resolution was insufficient to disentangle
more settle factors such as specific 2-joint motif designs. For this reason, we
designed new PDMS micro structures that systematically investigate axon guidance
primitives. Due to many group variants, low sample sizes, and high variance,
single comparisons were rarely significant after multiple testing correction.
Still, these experiments revealed interesting trends that enabled us to make
predictions about successful followup designs. The following discusses the above
in more detail.

\subsection{Growth cone tracking}
The implemented growth cone tracking model follows the common approach of
splitting the tracking problem into instance detection and identity association.
On test data, the model achieves detection F1 scores of 0.76. Subsequent
identity association reduces performances marginally by 0.05 to 0.71.
Intuitively speaking, 7 out of 10 growth cone identifications will be correct, 3
will either be false positives, false negatives or wrongly associated. \\

The tracker shows good generalization from train,- to test datasets, however,
meaningful model performance evaluation needs to incorporate generalization from
test,- to inference data as well. To expose the model to various growth cone
location contexts, the model was trained on three different PDMS micro structure
designs. This showed vastly improved generalization versus training on datasets
with a single PDMS micro structure design. Based on extensive qualitative
checking, with minor exceptions, we can report a solid overall generalization
ability to various PDMS micro structure designs. These exceptions include
out-of-focus timelapse recordings, and, limited to one experiment, detection of
growth cones in 1.5, and 3 $\upmu$m wide channels. We also observe that growth
cones are tracked less well in structures that are rarely present in the
training data, including rescue loops and the output channel. As commonly
observed in deep learning models \parencite{cnn_generalization}, the method is
highly sensitive to
deviating pixel intensity distributions. For new datasets, it is therefore
essential to carefully replicate the confocal image acquisition settings. \\

Not many solutions exist for the specific problem of growth cone tracking.
Relying on classical computer vision tools, \cite{gc_tracking} presented a
method that requires human assistance and clearly separated axons for successful
growth cone tracking. This early method did not fullfil our requirements in
terms of experimental throughput. An indirect approach inferring axonal growth
patterns from segmented and associated video frames also showed results
restricted to predominantly well separated axons \parencite{seg_track}. Compared
to a classical \textit{in vitro} setup, segmenting single axons in densely
packed PDMS micro channels is additionally challenging, since axons show heavy
overlap. Also, \cite{seg_track} did not publish an implementation for
open-source use. Another alternative relying on segmentation based tracking are
commercial solutions such as Incucyte by Sartorius, but they likely suffer from
the same problem. Classical non-machine learning based particle tracking tools
bundled in the Imaris imaging software (Oxford Instruments) did non perform
comparable to our method either. Another view on tracking growth cones in
timelapse image sequences is to treat the temporal dimension as a spatial one.
Targeting mainly connectomics, extensive work exists on axon segmentation in 3D
microscopy slices (\cite{3D_seg_1}, \cite{3D_seg_2}), however the z-dimension in
these datasets does not comprehensively translate to the temporal dimension in
video tracking. The specific objects of interest, growth cones, are rarely
present in 3D microscopy slices. For the reasons mentioned above, benchmarking
our method is not straight forward. Since, to our knowledge, no MOTA performance
scores have been reported on growth cone tracking, we looked into the reported
MOTA scores on the cell tracking with mitosis detection (CTMC) dataset
\parencite{organelltracking}. Here, state of the art performance reaches MOTA
scores of 0.5, whereas our tracker has a MOTA of 0.61. Of course, this may be
due to the CMTC dataset being far more challenging, but still, this crude
comparison indicates at least decent performance of our tracker. \\

A collection of model modifications could be made to potentially improve
tracking performance. First, the implemented YOLO detector would likely be
outperformed by the well established Faster R-CNN architecture
(\cite{fasterrcnn} and Introduction). The YOLO detecter was implemented because
of its wide adoption, but the advantage of fast online detection is not a
relevant feature in our offline detection use case. The model performance would
also likely increase with more training data including morphological growth cone
outliers and less common PDMS design motifs. Lastly, for long timelapse
recordings, the \verb|PyTorch| based video leading needs to be reimplemented as
currently, the entire image sequence is put in system memory.

\subsection{Directional PDMS micro structures}
The degree of directional growth through PDMS micro structures was screened by
growth cone tracking in 21 many-to-one designs, and by performing axon growth
experiments investigating lower level guidance primitives. While the tracking in
multiple-source to single-target PDMS micro structures aimed to find the best
performing design, the investigation of axon guidance primitives took a
conceptual step back, answering specific design optimization questions. \\

\subsubsection{Micro channel width}
Besides information on directional growth, tracking provided insights into axon
growth velocity and outgrowth frequency, revealing a strong relationship between
channel width and axon growth velocity. By reducing the channel width from 8 to
3 or 1.5 $\upmu$m, we observe an increase in median growth velocity from 750 to
1600 $\upmu$m per day (DIV 3.5-6.5). For aggregated cortical neurons grown in
agarose microcolumns, \cite{cullenrecent} presented similar growth velocities
peaking at DIV 3. Classifying stagnating axons in PDMS micro structures revealed
that designs utilizing narrow channels showed significantly fewer stagnating
axons than wide-channel counter parts. Based on the observation that growth
stagnation mostly occurs during the first 200 $\upmu$m, one hypothesis could be
that dentritic structures selectively enter and cloak channels wider than 3
$\upmu$m. \\
These high velocities do not necessarily lead to the conclusion, that the
majority of axons will cover distances of 3 mm in 2-4 days. First, it should be
considered the growth velocity is likely to drop over time as shown in
\cite{cullenrecent}. Second, fast, short living axon identities may achieve
growth velocities beyond 2 mm per day, yet only cover a few hundred $\upmu$m. On
top of that, similar to the discrepancy between 8 and 1.5 $\upmu$m channels,
growth velocity declines when axons enter the 75 x 75 $\upmu$m output channel.
For these reasons, we only see RGCs reaching the 3-4 mm distant target after
approximately 12 days (see Figure \ref{I_thal_innervation}). \\
Through the strong forward convergence, Design 3 implementing 3 $\upmu$m wide
channels achieved $\delta$ fold directionality above one, indicating a positive
effect of smaller channels on directionality. While these narrow channel designs
could be highly interesting for guidance motifs relying on high growth
velocities, the long time neural viability and signal transmission efficacy
would first need to be validated. \\

\subsubsection{Merging channels}
Any design that aims to guide multiple source nodes towards a common target node
inevitably needs to merge channels. Crucially, these merging elements also
present the opportunity for axons to cross-grow towards a neighboring node.
Thus, achieving directional merging is the central problem to solve for
engineering directional many-to-one PDMS micro structures. \\
Our 21 designs broadly implement three approaches to merging: (i) merging
channels using 2-joints, (ii) merging channels in a final lane, and the outlier
approach (iii), collecting axons in a large container. Design 00 solely follows
approach (iii), Designs 01-04 rely only on (ii) and the Designs 05-20 implement
both (i) and (ii). An illustration of 2-joints and final lanes is shown in
Figure \ref{R_designs}. \\
Guiding channels in an axon collection container with a diode-like geometry
(iii) does not yield high directionality, as seen in Design 00. Due to missing
edges, axons entering the container compartment slow down significantly and
frequently enter neighboring nodes regardless of the diode-like entry geometry
(see \verb|D00_tracking.mp4|). \\
In contrast, the addition of 2-joints significantly reduces the degree of
backwards growth in PDMS designs, albeit that no significant difference is
observed between the use of one or three 2-joints. Additionally, placing them
early, and thus closer to source nodes yields significantly higher forward
growth. This positive effect may be due to neighboring nodes emitting growth
promoting chemical cues. While the addition of at least one 2-joints
significantly reduces backwards growth, it seems to negatively affect the
frequency and velocity of axonal outgrowth. Designs omitting 2-joints exhibit a
median number of 65 axons per micro structure half, while designs with one or
three 2-joints, showed only 50 and 40 axons, respectively. This may be caused by
a thalamic cue diffusion bottleneck introduced by joining lanes together. The
growth velocity may be reduced due to deceased edge attachment within
2-joints.\\

As mentioned above, we do not observe a significant directionality difference
between designs implementing one or three 2-joints. However, assuming that final
lane merging (ii) does not outperform 2-joint merging (i) by a large degree,
there exists a theoretical advantage of using 2-joints over final lane joints,
especially when scaling up the number of source nodes. As a function of the
number of source nodes $x$, consider the integer $n(x)$ referring to the number
of directional axon transitions needed to obtain a completely directional
culture. For example, following approach (i), four source nodes merged with
three stacked 2-joints evaluate to $n_{(i)}(4) = 2 + 2 + 2 + 2 = 8$, since each
source channel passes two growth decision points. Interestingly, for an
increasing number of source nodes $x$, designs joining through final lanes
require vastly more favorable axon transitions $n_{(ii)}$ than designs relying
on 2-joints $n_{(i)}$ (see equation \ref{i_merging}, \ref{ii_merging}, and
$x=32$). Final lanes may still be competitive if merging directionality is
vastly higher than in 2-joints, however, since we do not observe exceptional
directionality in designs relying solely on final lane merging, we may assume
that 2-joints merge at least as directional as final lanes. Final lane joining
seems to positively impact axonal outgrowth frequency and velocity, but for
maximally directional cultures integrating many source nodes, stacked 2-joints
are likely preferred. The Designs 06 and 08 integrating three stacked 2-joints
did perhaps not outperform one 2-joint counter parts, because they did not
integrate rescue loops. \\

\vspace{5mm}
\hspace{1cm}
\begin{minipage}{0.3\textwidth}
    \hspace{5cm}
    \begin{tabular}{@{}rrrrr@{}}
        \toprule
        $x$ && 8 & 16 & 32 \\
        \midrule
        \vspace{2mm}
        $n_{(i)}(x)$ && 24 & 64 & 160 \\
        $n_{(ii)}(x)$ && 35 & 135 & 527 \\
    \end{tabular}
% \caption[Example]{Example}
\end{minipage}%
\hfill
\begin{minipage}{0.6\textwidth}
    \begin{equation}
        n_{(i)}(x) = x \log_2{x} \hspace{4.1cm}
        \label{i_merging}
    \end{equation}

    \begin{equation}
        n_{(ii)}(x) = x-1 + \sum_{j=1}^{x-1} j \quad \approx \frac{1}{2}x^2 \hspace{1.5cm}
        \label{ii_merging}
    \end{equation}
\end{minipage}%
\vspace{5mm}

An intuitive next step for improving directional merging was to investigate
optimal 2-joint merging designs. Since directionality performance of Designs
12-15 did not differ significantly, we designed a PDMS micro structure
specifically testing various 2-joint designs. Although the Kruskal-Wallis test
identified the 2-joint designs as originating from different distributions, only
a design implementing angled joint entries significantly underperformed versus
the control. The control joint showed a median merging bias $\beta$ of 0.6,
whereas we expect $\beta$ to be 0 for non-directional 2-joint designs. This
surprisingly high performance raised the bar for reaching significance. The
non-significant trends of merging performance we see in these primitive
experiments and the Designs 12-15 do not coincide. Wheres tangential joint
entries, doubling the 2-joint size, and 3 $\upmu$m joint entries perform well in
the isolated 2-joint screen, Designs 13 and 14 using 2-joints with inlays and 5
$\upmu$m entries respectively tended to be more directional in tracking
experiments. One issue reducing the experimental resolution of the primitive
screen was that successfully merged axons occasionally turned 180 degrees within
the diffusion well, and grew backwards into the 2-joint, therefore confounding
the measured directionality. To avoid this and find significant differences,
more replicates are required at a preferably early timepoint such as DIV 7. 

\subsubsection{Grwoth direction rescue loops}
% mention stomach
% angled, straight
% los res in primitive

The evaluation of $n_{(i)}(x)$ for 32 source nodes aptly highlights the
difficulty of engineering directional many-to-one PDMS micro structures
(equation \ref{i_merging}). Even when merging with 99\% directionality, the
joint event of entirely correct transitions within a culture occurs only with a
probability of $0.99^{160}=0.2$. One approach to alleviate the impact of merging
is to introduce rescue structures that selectively redirect wrongly growing
axons. Thereby, transitions within a 2-joint can be mitigated by preventing
axons from reaching neighboring nodes. As mentioned in the introduction, a
multitude of PDMS geometries have been published to impose growth
directionality. Stomach structures showed directionality of up to 95\% for
connecting two nodes \parencite{forro}, however their design is not directly
transferable to multiple source nodes converging on a common target. For this
reason, we designed new growth-rescuing loop structures that are small and
easily stackable. Importantly, the forward growth is not impaired, but even
increased in designs implementing three instead of one or zero rescue loops.
Although we observe lower axonal growth velocity when three of these rescue
loops are added prior to merging, the relative number of backwards growing axons
is significantly reduced. One caveat to this is that, although not statistically
significant, one rescue loop designs seem to be less directional and promote
faster growth than zero rescue loop designs. Analyzing the feature of \textit{n
rescue loops} as a numerical,- instead of categorical variable may result in a
different conclusion. \\

To investigate settle design variations between rescue loops, and to compare
them to previously published stomach structures, we performed an additional
experiment specifically targeting directional growth. The setup consisted of two
GFP, and RFP labelled nodes connected through channels with,- and without
directionality imposing motifs. Similar to 2-joint comparisons, the results were
impaired by many axons growing a loop from the target back to the source node,
using the non-directional control channels. Discerning if a GFP labelled axons
originated from the source or target node was thus often not possible. To
prevent an over,- and underestimation of forward and backwards growth
respectively, a second iteration of this experiment should only test a single
design between two nodes. \\
Though not significant, we see that stomachs successfully prevent backwards
growth, angled rescue loops tend to be more directional than straight ones, and
forward growth is unimpaired across the board. Between Design 12 using angled
rescue loops and Design 16 using straight ones, no significant difference is
observed.

\subsubsection{Bias of between dataset variance}
One of the most dominant trends in the axon tracking data was found between
Datasets 3 and 4, which originated from two independently conducted experiments.
Dataset 4 showed significantly less frequent axonal outgrowth, lower growth
velocity, and more stagnating axons. On top of this, due to the majority of
axons not reaching critical merging regions, the directionality performance
differed only marginally between designs. This resulted in an overestimation of
directionality scores when only the growth direction was evaluated. To address
this bias, we additionally looked into the fold directionality defined as the
ratio of target,- and neighbor reaching axons. Here, samples from Dataset 4
showed significantly lower performance than those of Dataset 3. \\
This discrepancy needs to be discussed because a subset of designs were only
included in one of the two datasets. Designs 03 and 04 using narrower channels
were dropped from Dataset 4 due to poor tracking accuracy. Importantly, the
relationship between narrow channels and increased growth velocity was
qualitatively observed in Dataset 4 as well. Designs present only in Dataset 4
included 00, 08, and 20. For these designs, the dataset-wide bias on for example
growth velocity and directionality needs to be carefully considered when
interpreting the results. The negative negative relationship between number of
2-joints and growth velocity is one such example. This trend might be
biased since one of the two three 2-joint designs, Design 08, carries the low
growth velocity bias present in Dataset 4. \\

A multitude of factors might have caused diminished axonal viability in Dataset
4. PDL or Laminin coatings may have been of inferior quality, the retinal tissue
might have stayed in Hibernate medium for too long, or problems during
dissociation could have harmed the RGCs. Another hypothesis is that the slightly
differing experimental setups between the two experiments caused the significant
deviations. While Dataset 3 used free floating thalamic tissue pieces enclosed
by a PDMS frame for chemical attraction, in Dataset 4, 3 mm distant stomach
target structures were seeded with a thalamus spheroid. This may have resulted
in significantly lower attractor cue concentration. 

\subsubsection{Axon guidance primitives}
% for the above, you treat numerical variables as categoric ones
Both merging and rescue loop motif designs are based on axon growth bias towards
edge attachment and the avoidance of sharp turning. However the details of these
growth biases are poorly characterized. To find elucidate these primitives of
axon guidance, we performed a range of experiments. Addressing chemical
guidance, we designed PDMS micro structures integrating growth decisions points
with varying channel distances towards a chemical attractor. Even when one
channel was 6x longer than the other, we did find biased growth into the shorter
channel. It is conceivable that too little thalamic attractor entered channels
in general, creating negligible cue gradients at the decision point.
Also, the experimental setup may be flawed because, at chemical equilibrium, the
growth decision region may not exhibit a differential attractor cue gradient
between the two channels. Since the cultures were imaged at DIV 7 and 14, the
chemical gradient may have been insufficient. To avoid the necessity of axons
reaching growth decision points before chemical equilibrium, a new experiment
could investigate the growth bias between different tissues. Such a micro
structure would more closely resemble final lane joints where thalamic attractor
concentration is expected to be higher than competing cues emitted by
neighboring RGC source nodes. \\

Besides experiments on chemical guidance, we tested the impact of specific PDMS
geometries on mechanical axon guidance, specifically on edge attachment and edge
transitioning. For concave edge transition motifs, both the radius r and opening
diameter d were identified as significant factors for guiding axons from one
edge to another. We see a non-significant trend indicating the benefit of large
radii and small opening diameters of 2 $\upmu$m. While 2-joints integrating 1.5
$\upmu$m wide entries did not achieve particuarly high directionality, small
entries into a wide final lane as used by Design 20 resulted in the highest fold
directionality out of all designs (see \verb|D20_tracking.mp4|). \\
Instead of integrating concave detachment from an unpreferred edge, we also
considered convex or radial detachment utilized i.a. in stomach structures.
Surprisingly, we don't find a relation between small radii and an edge
detachment bias promoting directional growth. Considering that only few axons
reached the decision area in these micro structures, the results may be due to
an insufficient sample size. \\
Lastly, we considered biasing the growth on the preferred edge a priori by
adding spike,- or hill like features to the unpreferred edge. This was found to
be a significant factor effecting the growth bias at a decision point. However,
again, single comparisons against the control with Holm-Bonferroni correction
were not significant. Using hill-like features with a radius of 4 $\upmu$m, or
spike features with a radius of 6 $\upmu$m showed a tendency towards biasing
axonal growth. Design 18 implementing spiky tracks did not perform significantly
better than designs omitting spikes. This result is in accordance with the
primitive experiment, since Design 18 implemented spikes of radius 2-3 $\upmu$m.
These features did likely did not resolve during micro fabrication, and
therefore failed to bias growth direction.

\subsection{Conclusion \& Outlook}
In this work, we screened 21 many-to-one PDMS micro structure designs for
directional growth towards an output channel. To do this at high-throughput, we
build a machine learning based growth cone tracking model enabling us to
identify correctly,- and incorrectly growing axons. The best performing design
exhibited a median fold directionality of almost 4, which translates into this
design exhibiting 4 times as many axons in the output channel, as in neighboring
RGC wells. To further optimize design primitives such as merging structures and
rescue loops, we designed followup experiments investigating axon guidance at a
lower abstraction level. While these experiments did not reveal clearly optimal
2-joint designs, they indicated the advantage of spikes on unpreferred edges and
small between-edge distances when transitioning to preferred edges. \\
Using Design 20 showing a directionality bias of nearly 4 fold, ensuing
iterations of biohybrid neural implants will benefit from fast convergence
towards the final lane while cross growth to neighboring seeding wells is
minimized. \\

A crucially important followup to this work is the functional confirmation of
directional connectivity within PDMS micro structure designs. While anatomical
directionality screens have the advantage of high throughput, the final metric
of interest is the number of functionally independent electrical channels
driving activity in the target tissue. Since all of the 21 micro structures were
designed to fit electrode patterns on glass multi electrode arrays (glass
MEAs), this data can be acquired without difficulty. \\

Although a subset of the 21 designs showed already impressive directionality,
there is likely room for further improvement. Future designs should integrate
narrow channels for higher growth velocities and potential benefits on channel
merging. Primitive experiments testing the performance of 2-joints should be
repeated to identify significant performance differences. Using optimal
2-joints, new designs should try to integrate as many angled rescue loops as
possible. Further, the potential advantage of using closely positioned thalamic
tissue pieces as chemical attractors should be investigated more rigorously. \\

Lastly, another interesting avenue of future research could be a
super-high-resolution screen of axonal growth behavior, using the growth cone
tracking model. This could reveal exact dynamics of edge transitioning and
attachment, growth cone stagnation and collapse, and turning,- retraction and
exploration behavior. From this low level knowledge, we could design PDMS micro
structures specifically tailored to growth cone dynamics. Growth cone tracking
at scale could also be used for studying branching behavior and target tissue
innervation.
\thispagestyle{plain}
\begin{center}
    \section{Abstract}
    \vspace{0.4cm}
\end{center}

Conventional brain computer interfaces of the last decades have relied on metal
electrodes to interface biological neural systems. Whilst advances in electrode
technology have enabled an exponential rise in single-neuron recording
bandwidth, progress on the ability to electrically stimulate multiple single
neurons \textit{in vivo} has been stagnating. To transition neuroscience from
observing correlations to imposing neural activity at high bandwidth and
spatiotemporal resolution, capable stimulation technology is indispensable.
Inspired by \textit{in vitro} neuroscience and tissue engineering, biohybrid
technology presents a promising paradigm shift for potent long-term
biocompatible neural stimulation. This work revolves around a biohybrid implant
utilizing ectopically grown retinal ganglion cells (RGC) in a microfluidic
system as living implantable electrodes. To enable high bandwidth and
functionally independent stimulation channels, this work addresses directional
RGC growth through our Polydimethylsiloxane (PDMS)-based micro structure
designs. Using a machine learning based growth cone tracking model on confocal
recordings of axonal outgrowth, we tested 21 micro structure designs for their
directional convergence from RGC seeding nodes into the implantable output
channel. We successfully identified micro structure designs that exhibited 4-2.5
times as many axons reaching the output channel versus neighboring nodes. On top
of that, we conducted followup experiments specifically investigating axon
guidance primitives and the effect of design paradigms on achieving
uni-directionally connected cultures. These insights on engineering directional
living neural networks may not only yield higher bandwidth biohybrid interfaces,
but prove useful for functionally more sophisticated brain-on-chip models. 
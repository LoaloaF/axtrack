\section{Preface}
Understanding the human brain requires neuroscience to develop complexity
reducing model systems that capture relevant functional, anatomical or chemical
features. The evaluation of which abstraction level and thus model system is
appropriate for answering key functional questions about the brain has long been
a source of controversy. This discussion is fundamentally rooted in the tension
between losing essential features in overly simplified model systems, and
dealing with overwhelming complexity and low experimental throughput in model
systems more closely resembling the human brain. \\
Although this work broadly employs \textit{in vitro} model systems thus trading
off resemblance to real brains for higher throughput and reduced complexity, it
is not primarily motivated by a certainty that this high abstraction level will
indeed \textit{solve} fundamental neuroscientific questions. Instead, this
project aims to follow an approach that has led to notable progress in other
domains, most prominently artificial intelligence: Developing an understanding
of the system by engineering it. This method has found adoption in domains as
neuromorphic engineering \parencite{neuromorphic} and neural engineering
\parencite{neuroengineering}, with the latter focusing i.a. on building
technology from living neural systems. Guided by the engineering problems, the
hope of these domains is that relevant neuroscientific questions are answered
along the way. And even if this does not come true, advancement in these fields
may still result in useful new technologies. The science presented here follows
this pursuit.\\ 

% You can only do engineering at that scale. that's why were at that scale. DId
% you make that clear enough? Inconsistency:  you mention this here in the
% beginning but this idea is not really found thoughout the introduction.
% Motivation is put on bad stimulation ability, and bad in vitro network models. 
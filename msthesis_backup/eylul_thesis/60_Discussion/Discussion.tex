\chapter{Discussion}
\label{ch:Discussion}


 
\subsection{Significance and Interpretation of Results}

\quad Overall, the PDMS:Hexane glue dilution ratio of 1:40 found to be optimal for gluing the 6$\mu$m PDMS microstructures on top of the stretchable MEAs without clogging the channels. The t-peel test revealed that the 1:40 glue has a strain energy density as strong as thin film of uncured PDMS glue. On the other hand, PDL and laminin coating was found to be the most effective for promoting axonal growth based on $\%$ filled channels with axons, $\%$ surface area covered with axonal growth and distance travelled in the output channel. PDL and laminin coating by desiccation was found to be the best coating strategy for the micro-structure-MEA assembly which yielded $\%$ filled channels and $\%$ surface area covered with axonal growth similar to that of the Control PDL+laminin group. Finally, a procedure to precisely align, glue and seal PDMS microstructures on top of the electrode tracks using a custom set-up was developed. The final proof-of-concept stretchable device that resembles the conventional MEA systems can be used to perform stimulation experiments. 

\quad According to the results, 1:40 dilution was found to be the optimum as a trade-off between having a sufficient amount of PDMS in the glue dilution for stronger bonding and avoiding clogging of the channels. Moreover, there was no axonal escape observed in the glued structures. Reproducibility of the glue dilution was further evaluated by performing cell culture experiments which showed that out of 60, 58.95 $\pm$1.40 of the microchannels were open whereas 4.44 $\pm$4.56 of the junction channels were clogged. Overall, the 1:40 glue dilution yielded reproducible results although there were a small number of clogged channels which can potentially be explained by the glue re-flowing in to the channels due to capillary forces. In order to overcome this issue of glue flowing into the channels, the wafer can be pre-baked at a low temperature to increase viscosity before the stamping process. \cite{yang2019microfluidic}) Alternatively, cloggings can be a result of manual mounting which relies on steady hands and dexterity as well as the cleanliness of the wafer which suggests that more reliable gluing can potentially be achieved by using the mask aligner set-up in a clean room setting. The slight variations observed could potentially also be due to errors while pipetting viscous PDMS. Diluted PDMS glue is considered as an effective bonding compared to O$_{2}$ plasma bonding. Firstly, micro-channels assembled using $\mu$TA/glue have been previously reported to have more defined boundaries where the PDMS prepolymer can fill nanogratings on the rough surfaces providing an effective sealing thus preventing leakage. On the other hand, performance of the glue/$\mu$TA is more robust against contaminants such as dust whereas presence of dust can fail an O$_{2}$ plasma assembly. \cite{yang2019microfluidic} Another drawback of O$_{2}$ plasma bonding is the reproducibility and reliability. \cite{eddings2008determining} It's been previously reported that bonding strength of of O$_{2}$ plasma assemblies can be subject to fluctuations up to 50 $\%$. \cite{koh2012quantitative} \cite{duffy1998rapid} These arguments are also indicated in the results of the t-peel test where although the plasma bonding was stronger than the bonding with 1:40 PDSM:Hexane glue in one case, the other two cases failed leading to a strain energy density of 0. Also, glue/$\mu$TA technique can be performed using only a spin coater whereas O$_{2}$ plasma relies on a plasma generator. \cite{yang2019microfluidic} Last but not least, performing plasma with the proposed alignment method may not be practical to implement since the plasma activated surfaces should be brought together as quickly as possible. On the other hand, the glue can further be improved in a number of ways. It's been previously reported in a number of studies that hexane can be used to dilute uncured PDMS in order to spin thin films. \cite{koschwanez2009thin} However, hexane is known to swell cured PDMS. This can be avoided by using a longer spin coating time (>5mins, although data not widely available) or dilutions can be performed using tert-bulyl alcohol which has been shown to be biocompatible. Alternatively, UV-PDMS after different exposure times can be tested. 
 
\quad No axonal growth was observed in the control PDMS group compared to the other treated groups which is most likely because of low cell attachment on the surface indicating that surface modifications are necessary for improved axonal growth inside the micro-channels as previously reported \cite{sahebalzamani2017enhancement}. Although PDMS is an excellent material for microfluidics and stretchable electronics, the presence of organic methyl groups present in the chemical composition of PDMS makes the surface hydrophobic leading to poor wettability and weak cell attachment. Therefore, surface coating is indispensible to promote optimal cell adhesion by increasing hydrophilicity. Poor adhesion in turn leads to insufficient neurite outgrowth \cite{\cite{akther2020surface}}. Based on the parameters assessed such as $\%$ filled channels, $\%$ growth area and distance travelled in the output channel to evaluate the success of the coating condition, control group coated with PDL+Laminin yielded considerably more growth at a faster rate compared to the Control PDL group. This finding is consistent with the results from the literature. \cite{sun2012surface} PDL and laminin coating has been widely used in neural cell cultures and scaffolds in numerous studies. \cite{orlowska2018effect} The difference observed between PDL and PDL+Laminin groups could be due to the fact that, although neural adhesion can be faster with more neurites on PDL coated surfaces, it's been shown that laminin can facilitate neuritogenesis more, therefore, can promote axonal guidance and growth of longer neurites and axonal guidance. \cite{orlowska2018effect} \cite{sun2012surface} In more detail, PDL is a positively charged synthetic polymer which can bind to the negatively charged PDMS. However, less biological activity have been reported on synthetic polymers as confirmed by the results of this study. \cite{jemni2020first} Polycations such as PDL are commonly combined with proteins such as laminin which is one of the typical components of the ECM help mimic the native ECM structure of the host as closely as possible and act as a biochemical cue to promote cell adhesion. \cite{sahebalzamani2017enhancement} \cite{orlowska2018effect} \cite{jemni2020first} By this way, cells can adhere to laminin though their membrane receptors. Laminin has also been shown effective in inducing cell attachment in scaffolds as well. \cite{orlowska2018effect} For example, Peng et al (2016) has reported a bio-scaffold coated with laminin for RPE implantation where the macular function was preserved up to 2 years in vivo  \cite{peng2016laminin} \cite{rochford2020bio}. Therefore, two component coating with PDL and Laminin was chosen as the optimal coating for a favourable environment for cell adhesion and viability in the micro-channels. 

\quad Coating with PDL and laminin by desiccation after mounting of the structures yielded axonal growth similar to that observed to the PDL and laminin on glass control group in terms of $\%$ channels filled and $\%$ surface area covered by the axons. Moreover, control PDL+laminin on glass and coating by desiccation yielded reproducible results across different experiments and structures where axonal growth in each group was similar. However, distance travelled by the axons after coating by desiccation was significantly lower compared to the control PDL and laminin group which will be further discussed in the limitations of the study.

\quad Substantially low growth was observed in the PDL on PDMS group. In addition, other groups did not facilitate axonal growth as much as in the control and coating by dessciation groups which can potentially be explained by the fact that coating can be affected by the heat or the glue flowing on to the surroundings and damaging the coating. Moreover, interaction between the coating agent and hexane as well as uncured PDMS is not known which can also potentially have a negative effect on the coating.

\quad The observed decrease in the distance travelled in the output channel by the axons in the non-plasma groups after Day 22 indicate collapsing of the bundle meaning the cell attachment is unstable and not sustainable over long-term. This formation of an unstable bundle that collapses around after Day 22 in no plasma activation coatings whereas a distributed growth in the output channel in the plasma treated cases was consistently observed. This can potentially be explained by the fact that O$_{2}$ plasma treatment results in hydrophobic hydrocarbon groups (-CH$_{3}$) being replaced by hydrophilic silanol groups (SiOH) after oxidisation thus improving wettability. This increase in wettability in turn facilitates the adhesion of the coating material for a more sustainable cell attachment \cite{akther2020surface}. The downside of using plasma treatment is the hydrophobic recovery over time, which may not be suitable for long-term cultures. In addition, it should be noted that there has been a report where laminin when used together with oxygen plasma was less effective than other ECM proteins \cite{akther2020surface}. On the other hand, toluene washes were omitted from the protocol, as the swelling of the PDMS microstructures in a solvent such as toluene can lead to the distorsion of the required  microchannel dimensions \cite{koh2012quantitative}.


\quad A strength of this study is the use of primary cell cultures which are harvested from healthy animal tissues and compose of naive cells which can provide a realistic model of axon regeneration and behaviour in vivo compared to single cells \cite{geuna2016vitro}. The variability observed within the same experimental groups can be explained by a number of reasons. Overall, marginal changes and variations observed in axonal growth within the same groups over time, could be due to alterations in the chemical microenvironment (i.e. pH) and digestion by cells which result in changes in or degradation of the biochemical coating \cite{blau2013cell}. More importantly, although explants have a higher survival rate \cite{jemni2020first}, cutting explants of similar size reproducibly is a challenge and the efficiency of the AAV transduction efficiency might be reduced since the virus needs to penetrate and infect the whole explant rather than single cells. Therefore, using spheroids for the future experiments is likely to yield more reproducible results and result in more uniform viral tagging of the cells.

\quad As indicated by the coating results, the heat treatment might have an adverse effect on the coating and slow down the axonal growth perhaps due to the denaturation of laminin at high degrees \cite{charonis1985binding}. Furthermore,as mentioned before, the interaction of the uncured glue with the coating is yet unknown. The glue might flow onto the coating which can be detrimental to the coating. Due to these reasons and in order to minimise the potential damage to or drying of the coating during the alignment and mounting using the mask aligner, coating of the electrodes was done by desiccation after the structure have been aligned and mounted which yielded a similar growth pattern to control PDL+Laminin group. This also allows more flexibility during the alignment process and handling of the electrodes and mounting of the rings afterwards. Furthermore, having the structures already been mounted and cured before the coating offers the possibility of curing the glue overnight at 80 $^{\circ}$C to make sure hexane between the two PDMS layers completely evaporates and the thin layer of glue fully cures.  Moreover, ethanol or acetone can be used for sterilisation of the devices or to remove uncrosslinked oligomers prior to coating and cell seeding without the risk of damaging the coating. More importantly, for the next stage of the project where a collagen tube will be incorporated to the end of the channel, it is essential to coat the microfluidic channels after the mounting process has been completed which might otherwise lead to drying of the collagen tube at high degrees. 

\quad The gluing technique makes it more challenging to perform alignment on the electrodes since moving the stamped structure on the electrode surface can lead to the glue being smeared around which in turn might clog the channels. The proposed alignment method makes it possible to perform alignment and seal the structure on top of the electrodes in one go, reducing the risk of channels being clogged by the glue being smeared around. Moreover, the method offers a precise and a more reproducible way of aligning since it's less dependent on dexterity and hand alignment thus minimising human error that can be caused by unsteady hand motions. Overall, the feasibility of the proposed assembly method as well as the biocompatibility of microstructure-MEA assembly was confirmed by culturing cortical nerves on top of the electrodes. Weights were added to apply a slight pressure to make sure that there are no visible voids due to variable surface flatness of the MEA surface.

\quad Explanation for the difference between nanowire Au and Pt could be that the surface roughness migth be different. The exposed parts of the nanowires embedded in PDMS can act as nano topography cues and promote cell adhesion. Alternativel culture on Pt electrodes can be improved by pre-culturing the electrodes with cells to make sure anything that might be toxic for the cells has been cleared out and the second culture can yield better growth due to residual cell debris \cite{hong2017chip}. Moreover, the irreversible bonding and washing protocol using tergazyme makes it possible to re-use the stretchable MEAs instead of repeating the gluing and re-alignment process. This can make the stimulation experiments less time-consuming in the future and keep the alignment consistent with each experiment.

\subsection{Limitations and Future Work}

\quad The study has a number of limitations. First of all, all the coating experiments were performed with triplicates which reduces the statistical power considering the biological variation observed. On the other hand, a new challenge introduced by the proposed coating method, PDL+Laminin by desiccation after mounting,is that the surface of the PDMS microstructures is coated as well. This results in axons growing also on top of the surface instead of only inside the channels which can potentially explain why the distance travelled in the output channel in the groups coated by desiccation is lower compared to the control PDL+Laminin group. In order to circumvent this difficulty in the final device, an anti-fouling coating agent such as PMOXA \cite{weydert2017easy} or a mask on the surface of the PDMS microstructure can be used.  Another explanation could be due to the wash steps after the PDL coating. Despite 10 minutes wait in between every wash step and re-desiccation and incubation with laminin overnight, it might be likely that a small amount of the PDL solution remains inside the channels and doesn't diffuse out. As mentioned previously, any residual non-physisorbed PDL fragments trapped in the microchannels after the PDL washing procedure can be toxic to the cells. Moreover, as indicated by the results of the coating experiment, a higher intensity was measured at the beginning of the channel which can either be due to residual PDL left inside the channel or the beginning of the channel having a thicker PDL coating. This potential issue can systematically be checked by first coating the surface with normal PDL followed by coating with fluorescent PDL and performing a time lapse experiment to be able to determine the time it takes to completely wash off the fluorescent PDL from the microchannels. It is critical to first perform the coating using PDL in order to avoid fluorescent PDL coating the surface and make sure it remains in the microchannels so that the washing off can be tracked. Otherwise it becomes challenging to differentiate between the PDL coating and PDL inside the microchannels if fluorescent PDL is used directly. Another possible explanation could be the pH changes during the desiccation step affecting the coating. 

\quad There are also certain limitations associated with assessing the bonding strength measurements. T-peel test might not represent the actual forces the implant will be exposed to after implantation to the brain. This makes it challenging to compare the bonding numerically with literature to be able to accurately judge if the bonding is strong enough to endure the implantation and forces exerted through micro-motions after conforming on the brain without failing \cite{sridharan2013long}. Other tests that can be used to confirm the bonding strength and further characterisation in a scenario that better mimics and reproduces the forces in the brain include lap-shear tests and bulge tests on the assembly itself \cite{sofla2010pdms} \cite{thangawng2007ultra}. The bonding strength with the electrode surface should also be assesed directly on the electrode surface rather than plain PDMS. This is because the surface properties and roughness of the electrode surface so does the bonding could be different. Alternatively, the PDMS micro-structure can be filled with dye and the channels can be observed for leakage while the assembly is under cyclic tensile stretching or bending.

\quad There are also several features that can be improved in the design of the structures. In order to improve and speed up the alignment procedure, alignment marks can be added to the structures. Moreover, the micro-channels can be re-designed to be wider to facilitate more efficient diffusion and exchange of nutrients, make it possible to use a lower dilution ratio of PDMS without clogging the channels and to make the design more tolerable to misalignment.

\quad For the bundling analysis, although the Gaussian fit method can provide a numerical insight regarding the degree of bundling, it has a number of limitations. In cases, where there are more than one small bundles or when the axons are growing on the edges, the Gaussian fit becomes sub-optimal. In order to make the analysis more robust to such cases and biological variations, diffusion gradients can be calculated to evaluate axonal bundling where neuron tips are taken as diffusion sources in free space to derive the growth direction and elongation of neurites. This method is similar to diffusion tensor imaging which shows axonal bundling of fiber tracts in the real brain. \cite{marinov2020computational}. Alternatively, data can be fit using a multi Gaussian fit to better account for the presence of individual axon bundles and detect two separate peaks for the cases where the axons grow on the output channel walls. Last but not least, bundling analysis can be carried out using a point density function which can potentially identify the spreadness of growth specially in plasma cases more accurately. \cite{pajevic2013optimum}

\quad To further promote axonal outgrowth and neural adhesion, hydrogels can be incorporated into the microchannels. Hydrogel filling is commonly adapted in biomaterial-based scaffolds used in vivo axonal regeneration \cite{struzyna2017anatomically}. Annabi et al. previously reported a method for hydrogel coating of microfluidic channels using a microfluidic set-up where a photocroslinkable hydrogel was crosslinked under continuous flow inside the channels using UV light \cite{annabi2013hydrogel}. A similar strategy can be adapted in order to reinforce axonal growth inside the channels, particularly in the output channel, of the devices. On the other hand, although, axonal growth was confined inside the microchannels, axonal escape was observed around the edges of the electrode pads. 

\quad Although, alignment and cell confinement using the gluing and alignment described have been shown to be feasible and promising, the procedure should be tested with a larger number of microstructure-MEA assemblies in order to be able to derive statistical conclusions regarding the reproducibility of the proposed method. On the other hand, in order to further improve the cell confinement, the electrode pads can be better embedded by patterning selective openings through a photoresist lift-off technique as described by Park et al. \cite{park2009micropatterning}. Alternatively, weights lower than 5g can be used during the curing of the assembly to further seal the bonding between the electrode surface and the PDMS microstructure.

\quad The next step of the project will be to demonstrate that the MEA-microstructure composite electrode can be used to stimulate neurons. This can be done by electrical stimulation of the individual channels or wells and measuring neuronal activity using Calcium imaging. The alignment and tangency of the electrode pads to the microchannels or wells can also be optimised via stimulation experiments. The effect of coating thickness might need to be investigated in order to achieve a high signal to noise ratio during recordings or stimulations. 



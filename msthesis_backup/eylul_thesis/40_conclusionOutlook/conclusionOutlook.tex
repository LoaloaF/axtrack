\chapter{Conclusion and Outlook}
\label{ch:conclusionOutlook}



To conclude, PDMS was diluted in hexane in order to spin-coat a thin layer of glue. The optimal dilution that would reproducibly confine axonal growth without clogigng the channels was found to be 1:40 PDMS:Hexane. The ideal coating strategy was chosen to be PDL and laminin coating by dessication after the gluing of the PDMS microstructures on top of the electrodes. The coating still has a few minor challenges such as axons not travelling as far in the output channel or the surface of the structures being coated as well. In order to overcome such issues, the surface of the micro-structure can be masked during the coating or a micro-fluidic set-up can be used for the coating procedure. Last but not least, the feasibility of aligning and gluing of the PDMS micro-structures on top of MEA tracks using an alignment set-up has been demonstrated. In order to further improve the cell confinement around the electrodes, the electrode pads can be better embedded by patterning selective openings through a photoresist lift-off technique during the fabrication process. The final assembly of the stretchable microstructure-MEA prototype device that resembles the conventional MEAs will be used for the electrical stimulation experiments.
















%---------------------------------------------------------------------------------------------------

%=================================================================================
%---------------------------------------------------------------------------------
%\section{Limitations of the proposed modeling}
\label{ch:conclusionOutlook:sec:shortComings}

%=================================================================================
%---------------------------------------------------------------------------------
%\section{Recommendations for future work}
%\label{ch:conclusionOutlook:sec:recommend}
%\begin{itemize}
%	\item recommendation 1
%	\item recommendation 2
%	\item recommendation 3
%	\item recommendation 4
%\end{itemize}

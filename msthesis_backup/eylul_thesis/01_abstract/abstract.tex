\begin{center}\section*{Abstract}\end{center}
%

43.3 million people are affected by irreversible blindness worldwide as a result of conditions such as glaucoma, age-related macular degeneration (AMD), diabetic retinopathy and retinitis pigmentosa (RP). Stretchable electronics provide better conformal match with the implanted tissue thus less risk of rejection whereas biohybrid implants are a novel approach to house cultured cells in electronics in an attempt to improve biological integration. Both approaches are promising to fabricate implants that are stable in long term. The aim of this project to to assemble a stretchable biohybrid implant with axon guiding PDMS microstructures to achieve precise and selective stimulation of the dLGN tissue potentially restoring vision. Firstly, the dilution of the PDMS in hexane was optimised to obtain a thin film of glue in order to be able to bond the PDMS microstructures on top of the electrodes achieving refined axonal growth without clogging the micro-channels. Later, combinations of different coating strategies of PDMS were investigated to optimise the axonal growth in the final device. The metrics used to quantify the axonal growth are filling of the channels are with axons, surface area covered by the axons, how far the axons travel in the output channel and bundling. In the last step, a precise alignment method was developed in order to glue and seal the microstructures on top of the electrodes tracks. The optimal dilution that would reproducibly confine axonal growth without clogging the channels was found to be 1:40 PDMS:Hexane. Moreover, the ideal coating strategy was chosen to be PDL and laminin coating by desiccation after the gluing of the PDMS microstructures on top of the electrodes. The feasibility of aligning and gluing of the PDMS micro-structures on top of MEA tracks using an alignment set-up has been demonstrated. In order to further improve the cell confinement around the electrode pads, the electrode pads can be better embedded by patterning selective openings through a photoresist lift-off technique during the fabrication process. The final assembly of the stretchable microstructure-MEA device that resembles the conventional MEAs and will be used for the electrical stimulation experiments.




\clearpage
%
%====================================================================================================
\begin{center}\section*{Acknowledgments}\end{center}
\quad I would like to express my great gratitude to my supervisors Dr. Tobias Ruff, L$\'{e}$o Sifringer and my advisor Prof. Janos V\"or\"os for their significant guidance and advice during this highly multidisciplinary project, helping me to gain invaluable new skills and knowledge. I would also like to thank you for being always available to answer my questions and doubts as well as sharing their expertise knowledge in the fascinating and promising field of stretchable electronics and biohybrid implants which motivated me to further explore advances in these innovative research fields. Thank you Tobias for always being enthusiastic, giving me the opportunity to work on such an exciting project, all the discussions as well as all the new knowledge and skills I've learnt in this field. Thank you L$\'{e}$o for all the help and advice as well as teaching me the basics of microfabrication and working in a clean room. I would like to sincerely thank Janos for all the valuable input and discussions for the project as well as career advice.

\quad Thanks to Anna, Simon and Margherita for the team work, it was great working with you! Thanks Annina for being the best desk buddy and all the valuable input with image analysis. Jens and Stephan, thank you for all the advice, feedback and discussions about the project and image analysis, as well as trusting me with inkubot to take the cells on a road trip to ScopeM. Thank you Sophie and Ariana for the atto donations and valuable discussions! Also, thanks Sean for all the help with CLSM and practical suggestions. I would like to say thank you to all the members of the LBB family for an amazing 6 months in such a nice environment. I would also like to thank Dr. Tobias Schwarz for all the help and advice with imaging and the custom set-up. 

\quad Big thank you to my flatmates Charlotte, Claudia, Kai, Jacob, Johannes and Thibault as well as my friends Gizem, Augustina and Didem for all the encouragement and help! I would like to thank my cousins, Ilden, Elcan, Vural, Fatma, Jale and Laden for being inspiring role models. Lastly, I am extremely grateful to my family, my mother Gulter, my dad Ibrahim and my brother Ekin for their continuous support and help!
\clearpage
%
%====================================================================================================




